\documentclass[a4paper,11pt,titlepage,singlespacing]{article}
\usepackage[T1]{fontenc} 
\usepackage[utf8]{inputenc}
\usepackage[brazilian]{babel}

\usepackage{amsmath}
\usepackage{indentfirst}
\usepackage{graphicx}
\usepackage{hyperref}
\usepackage{multicol,lipsum}
\usepackage[bottom=2cm,top=2cm,left=2cm,right=2cm]{geometry}
%\usepackage{color}
\usepackage{cancel}
\usepackage{soulutf8}
\usepackage{color}

%%%% CALUÃ COMEÇOU A MEXER DAQUI %%%%%

\usepackage[easyscsl]{kpfonts}
\usepackage[T1]{fontenc}

\usepackage{floatrow}
{\renewcommand{\arraystretch}{1.25}
\usepackage{arydshln}

\newcommand\todo[1]{\textcolor{red}{#1}}

% \usepackage[hidelinks]{hyperref}
\usepackage[alf,versalete]{abntex2cite}

\renewcommand{\thesection}{\oldstylenums{\arabic{section}}}
\renewcommand{\thesubsection}{\oldstylenums{\arabic{section} · \arabic{subsection}}}
\renewcommand{\thepage}{\oldstylenums{\arabic{page}}}

\renewcommand{\yearstyle}{\oldstylenums}

\linespread{1.25}



\setlength{\topmargin}{-0.5cm} % extra vert. space + at the top of header: 23pt -15
\setlength{\oddsidemargin}{0.5cm} % extra spc added at the left of odd page: 0pt
% \setlength{\evensidemargin}{1cm} % ext. spc added at the left of even pg: 59pt
\setlength{\textheight}{23.5cm} % height of the body: 592pt
\setlength{\textwidth}{15cm}
\newcommand{\ignore}[1]{}


% https://tex.stackexchange.com/questions/255167/hanging-indent-in-headings
\usepackage{titlesec}
\newlength\TitleOverhang
\setlength\TitleOverhang{1.5cm}
\newcommand\Overhang[1]{%
  \llap{\makebox[\TitleOverhang][l]{#1}}%
}
\titleformat{\section}
  {\normalfont\Large\bfseries}{\Overhang{\thesection}}{0em}{}
\titleformat{\subsection}
  {\normalfont\large\bfseries}{\Overhang{\thesubsection}}{0em}{}
\titleformat{\subsubsection}
  {\normalfont\normalsize\bfseries}{\Overhang{\thesubsubsection}}{0em}{}






\begin{document}

% % % % % % % % % CAPA % % % % % % % % % %
\begin{titlepage}
	\begin{center}
		\large{Universidade Estadual de Campinas}\\
		\large{Faculdade de Engenharia Elétrica e de Computação}\\ 
		\large{Departamento de Engenharia de Computação e Automação Industrial}\\ 
        \vspace{100pt}
        %\textbf{\Large{Sistema de realidade virtual para apresentação de conteúdo na língua de sinais brasileira}}\\
        \huge\textit{A voz na letra: \\ avaliação de representações visuais \\ de prosódia em tipografia}\\
        
	\end{center}
	\vspace{80pt}
	\begin{flushleft}
        \large{\textbf{Pesquisadores Responsáveis:}}\\
        Profa. Dra. Paula Dornhofer Paro Costa (Orientadora)\\	
        Caluã de Lacerda Pataca (Mestrando)\\
        
        
        \vspace{10pt}
        \large{\textbf{Finalidade:} \\Tese de Mestrado}\\
        \vspace{10pt}
        \large{\textbf{Local:} \\
        Universidade Estadual de Campinas (Unicamp)
        \\Faculdade de Engenharia Elétrica e de Computação (\textsc{feec})}\\
        Depto. de Engenharia de Computação e Automação Industrial (\textsc{dca})\\
        (e ambiente web?)
        \vspace{15pt}        
	\end{flushleft}
	\vspace{3cm}
	\begin{center}
	 Data de Apresentação: Março, 2019
	\end{center}


\end{titlepage}

% % % % % % % % % % % % % % % % % % % % % % % % % % % % % % % % % %

% % % % % % % % % PRIMEIRA PÁGINA % % % % % % % % % % % % % % % % %
\begin{titlepage}
%\vspace{2cm}
% \begin{center}
\noindent \Large{\textbf{Dados dos Pesquisadores}}\\
% \end{center}

\vspace{10pt}
\begin{flushleft}
\large{\textbf{Nome}: Paula Dornhofer Paro Costa}\\
\large{\textbf{Função:} Professora}\\
\large{\textbf{Local de Trabalho:} Faculdade de Eng. Elétrica e de Computação (\textsc{feec}), Depto. de Eng. de Computação e Automação Industrial (\textsc{dca})}\\
\large{\textbf{Endereço:} \\Av. Albert Einstein, 400, Cidade Universitária ``Zeferino Vaz'', Distrito de Barão Geraldo, Campinas-SP, \textsc{cep} 13083-852}\\
\large{\textbf{Telefone de Contato:} (19) 3521-0246 }\\
\large{\textbf{Currículo Lattes:} \url{http://lattes.cnpq.br/4518009815956207} }\\
\end{flushleft}

\vspace{10pt}

\begin{flushleft}
\large{\textbf{Nome}: Caluã de Lacerda Pataca }\\
\large{\textbf{Função:} Aluno de Mestrado}\\
\large{\textbf{Local de Trabalho:} Faculdade de Eng. Elétrica e de Computação (\textsc{feec}), Depto. de Eng. de Computação e Automação Industrial (\textsc{dca})}\\
\large{\textbf{Endereço:} \\Av. Albert Einstein, 400, Cidade Universitária ``Zeferino Vaz'', Distrito de Barão Geraldo, Campinas-SP, \textsc{cep} 13083-852}\\
\large{\textbf{Telefone de Contato:} (19) 99289-1963 }\\
\large{\textbf{Currículo Lattes:} \url{http://lattes.cnpq.br/9013966339541270} }\\
\end{flushleft}
\vspace{10pt}

\vspace{10pt}

\end{titlepage}

\newpage

\abstract

\noindent A~leitura fluente se dá quando o leitor consegue relacionar internamente letras e símbolos com os sons da língua, que ele em geral já traz consigo, em uma espécie de \textit{fala} interna. Em certos contextos esse processo funciona mal — algumas crianças em processo de alfabetização penam para se tornar leitoras fluentes, falantes de língua estrangeira nem sempre mapeiam bem som a letra, surdos não têm as referências sonoras etc. Nosso projeto se insere no rol de pesquisas em tecnologia tipográfica que visam manipular a forma do texto de modo a complementar o conteúdo explícito na escrita com aspectos presentes na fala, investigando os impactos dessas intervenções sobre a leitura. Especificamente, construirá um algoritmo computacional que abstraia características acústicas da fala para mapeá-las visualmente em formas tipográficas, imbuindo no texto elementos expressivos da voz, em especial aqueles relacionados à emoção. Nossa hipótese é que na leitura de um texto composto com essa tipografia modulada pela voz haverá um aumento em medidas subjetivas como \textit{transporte} (perceber-se ``dentro'' da obra), \textit{identificação com as personagens} (empatia do espectador para com as personagens) e \textit{realismo percebido} (quão plausível parece ser uma narrativa). Juntas, essas medidas indicam a \textit{imersão} do leitor em uma dada obra. 

A avaliação se dará em ambiente web e compreende duas etapas. Na primeira, os participantes serão expostos a gravações contendo falas de atores e, ao mesmo tempo, a textos com as transcrições dessas falas e em cujas formas foi feita a manipulação da tipografia por nosso algoritmo de extração e representação de prosódia. Espera-se então que o participante relacione o áudio a essas manipulações no texto. Na segunda etapa, haverá um questionário online com perguntas objetivas e discursivas que buscarão mensurar o impacto dessas manipulações no texto sobre a percepção subjetiva de seu conteúdo.

\newpage


\thispagestyle{empty}
\newpage


\newpage

\section*{Relevância Social}

\noindent De acordo com o Censo Demográfico de 2010 \cite{censo2010}, quase 9,7 milhões de cidadãos Brasileiros possuem alguma perda auditiva permanente. Há evidências empíricas \cite{murphy-berman_impact_1983} de que para crianças surdas e alfabetizadas, a presença de \textit{closed captions} em programas de televisão se traduz em um entendimento mais aprofundado do conteúdo, em especial em relação a seu conteúdo \textit{afetivo}.

Em cima dessa constatação emerge que aperfeiçoamentos nas tecnologias de representação de texto, em especial através de técnicas que busquem representar qualidades expressivas exclusivamente presentes no registro sonoro — e, assim, pouco acessíveis à população com deficiências auditivas — podem aumentar a acessibilidade e fruição de mídias audiovisuais.

Mas os benefícios podem ser mais amplos: há também vasta literatura \cite{fiske_video_2015} documentando indícios empíricos de que as \textit{closed captions} trazem benefícios de compreensão, atenção e memória para diversos outros públicos: crianças, adultos e idosos; leitores experientes ou em fase de aprendizado; falantes ou não da língua em questão; etc. Assim, aperfeiçoamentos nessas legendas podem vir a ter benefícios de amplo impacto.

\todo{Paula, uma questão aqui: estou falando da possível importância do estudo para a população surda, mas *não* vou testar com surdos. Isso você acha que é um problema? Minha lógica, que pode ou não ter ficado clara no texto, é que as legendas são boas para qualquer público, o que significa que podemos testar com público em geral a princípio. }


\newpage


\thispagestyle{empty}
\newpage
\pagenumbering{arabic}

% % % % % % % % % % % % % % % % % % % % % % % % % % %

\renewcommand{\contentsname}{Sumário}
\tableofcontents

\newpage

%%%%%%%%%%
\section{Introdução}

\noindent Ler é uma habilidade cognitiva de alto nível, que exige longo período de treinamento e que envolve intenso processo neurológico. Dentre as estruturas cerebrais envolvidas no processo de leitura, é surpreendente notar que, além daquelas encarregadas do processamento de imagens, a leitura exige também o emprego das estruturas que lidam com processamento de sons \cite[cap.7]{seidenberg2017}. Isso ocorre porque, mesmo quando lê silenciosamente, cabe ao leitor deduzir (ou inventar) em sua voz interna qual é a musicalidade do texto, habilidade fundamentalmente relacionada à compreensão e interpretação do mesmo.

Investigar como essa ``voz'' emerge a partir do texto não se trata de uma questão meramente \textit{estética}. Certos tipos de dislexia, por exemplo, parecem antes causados por problemas nas estruturas neurológicas que processam sons do que deficiências no processamento de imagens, mesmo que se manifestem sob a forma de dificuldades na  leitura \cite[cap.8]{seidenberg2017}. Além disso, crianças em processo de alfabetização que leem de maneira monótona, ou seja, que não conseguem extrair do texto a expressividade da fala, tendem a desenvolver problemas de compreensão \cite{bessemans2017}. Finalmente — e ao contrário da noção vendida por certos cursos de leitura dinâmica de que uma leitura sem subvocalização traria ganhos de velocidade sem perdas na compreensão —, o leitor experiente se vale dessa voz interna para ter acesso à informação fonológica contida no texto e, assim, reduzir ambiguidades e facilitar a compreensão \cite[cap.4]{seidenberg2017}.

Neste contexto, o presente projeto se insere no âmbito da pesquisa em tecnologia tipográfica. Se debruça sobre algoritmos computacionais de transformação visual de texto almejando transformar o processo de aquisição de informação por meio da leitura mais intuitivo e acessível, com potenciais aplicações no auxílio ao processo de alfabetização, ensino de línguas estrangeiras e, entre ainda outras, como tecnologia assistiva, em especial para a população surda.

Em particular, o presente projeto propõe investigar e avaliar algoritmos de mapeamento automático de parâmetros acústicos da fala expressiva para parâmetros tipográficos do texto.



%%%%%%%%%%
\section{Objetivos}

\noindent Este projeto tem como objetivo propor, implementar e avaliar um modelo de mapeamento de características acústicas da voz, em especial aquelas relacionadas à expressão da prosódia, para modulações visuais em uma família tipográfica. 
São objetivos específicos do projeto:
\begin{itemize}
    \item avaliar se essas modulações permitem ao leitor inferir a presença de emoções no áudio que não estejam explicitamente presentes no texto;
    \item apurar que efeitos essa abordagem pode ter na imersão de um espectador que assiste a um filme em cujas legendas foram aplicadas essa modulações, calculadas a partir de propriedades acústicas extraídas da fala dos atores.
\end{itemize} 



% O objetivo principal desta pesquisa é avaliar a inteligibilidade de conteúdo sinalizado em Libras por um avatar, apresentado aos participantes da pesquisa.

% São objetivos específicos do projeto:
% \begin{itemize}
% \item avaliação da inteligibilidade de sinalizações em Libras pelo avatar com e sem ajustes pós-captura;
% \item avaliação da inteligibilidade de sinalizações em Libras pelo avatar em monitor convencional versus o ambiente de realidade virtual.
% \end{itemize}

%%%%%%%%%%
\section{Hipóteses}

\noindent A primeira hipótese a ser testada é que nossa abordagem para mapear a prosódia contida na voz em modificações visuais tipográficas produzirá resultados suficientemente auto-explicativos a ponto de que um participante da avaliação, não tendo recebido treinamento prévio, consiga identificar que um dado texto corresponda a uma dada enunciação sonora do mesmo.

A segunda hipótese a ser testada é que essas modificações causarão um aumento na imersão do participante em uma dada obra audiovisual, medida através do aumento em medidas subjetivas auto-reportadas como \textit{transporte}, \textit{identificação com as personagens} e \textit{realismo percebido}.

%%%%%%%%%%
\section{Local de Realização da Pesquisa}

\noindent As avaliações serão realizadas sob supervisão dos pesquisadores nas dependências do Laboratório do Departamento de Engenharia de Computação e Automação Industrial (\textsc{lca}), na Faculdade de Engenharia Elétrica e de Computação (\textsc{feec}), e em ambientes livremente escolhidos pelos participantes que optem por fazer o experimento online. 

Este segundo grupo terá recebido um convite por e-mail contendo uma apresentação do estudo, a indicação do período em que o teste estará disponível online, o perfil esperado para participação e os requisitos mínimos computacionais necessários para acessar o site. Ficam assim livres para escolher local e horário que melhor lhes convir para realização do experimento.

As avaliações presenciais do grupo de controle serão realizadas nas dependências do \textsc{lca}, na \textsc{feec}, em horários previamente agendados de comum acordo entre os pesquisadores responsáveis e o participante voluntário.

Os participantes serão informados durante o recrutamento e explicitamente via Termo de Consentimento Livre e Esclarecido (\textsc{tcle}) que o projeto não arcará com eventuais custos de deslocamento até o local da avaliação. Os agendamentos serão realizados de modo que coincidam com dia e horários de atendimento usual dos participantes pelo \textsc{lca}.

A sala onde será realizada os testes pode ser caracterizada como uma sala contendo computador, monitor e cadeiras e, no momento da realização dos testes estará dedicada à avaliação a fim de evitar distrações para os participantes e garantir a privacidade dos mesmos durante a avaliação.

%%%%%%%%%%
\section{População a ser estudada}

\noindent Serão realizadas chamadas de recrutamento para indivíduos fluentes em português, alfabetizados, maiores de idade e com acesso a um computador com sistema de som, internet e versões recentes dos navegadores mais comuns.

%%%%%%%%%%
\section{Garantias éticas aos participantes da pesquisa}
\noindent A metodologia empregada pelo estudo e seus procedimentos estarão descritos no Termo de Consentimento Livre e Esclarecido (\textsc{tcle}), em linguagem clara e acessível, o qual será apresentado a cada participante. Para os participantes que farão o teste sem supervisão, o experimento só se iniciará a partir do momento em que sinalizarem que leram e que concordam com o \textsc{tcle}, que estará disponível para leitura e \textit{download} no site do experimento. 

Para os participantes do grupo de controle, que farão o experimento sob supervisão em ambiente controlado, a anuência aos termos apresentados será registrada por meio da assinatura o termo em duas vias, sendo que uma via ficará em mãos dos pesquisadores e a outra com o participante.

%%%%%%%%%%


%%%%%%%%%%
\section{Metodologia Proposta}

\noindent Como já exposto, o experimento terá duas configurações: um primeiro grupo realizará o teste em ambiente controloado e sob supervisão dos pesquisadores; o segundo, em ambiente livre de acordo com a conveniência do participante, demandando apenas um computador com determinados requisitos.

\hl{Para os participantes que farão o experimento sob supervisão, antes do início do teste propriamente o \textsc{tcle} será apresentado verbalmente e em duas vias por escrito, que o participante deverá ler e assinar para prosseguir, ficando com uma das vias.}

\todo{Paula, estou imaginando que para o grupo de controle isso aqui teria que ser físico, mas se você achar que rola ser exclusivamente via site: melhor!}

As sessões supervisionadas ocorrerão em uma sala no Laboratório do Departamento de Engenharia de Computação e Automação Industrial (\textsc{lca}). O desenvolvedor estará no local, contudo não será apresentado para que não haja influência no julgamento dos participantes.

Nem os participantes do grupo sob supervisão tampouco os demais serão filmados ou fotografados durante o experimento.

Em ambos os casos, o experimento consiste no participante interagir com uma página web, seguindo os seguintes passos:

\begin{itemize}
    \item Apresentação resumida do projeto por escrito;
    \item Para os participantes que farão o experimento sem supervisão, apresentação do Termo de Consentimento Livre e Esclarecido (\textsc{tcle}). Para prosseguir com o experimento o participante deverá declarar ter lido e estar de acordo com os termos propostos;
    \item Questionário inicial para coleta de dados demográficos — idade, sexo (opcional), grau de formação, área de trabalho e/ou estudo (opcional). Para os participantes que farão o experimento sem supervisão, nesta etapa serão também coletados automaticamente dados sobre localização de acesso e informações sobre o computador e navegador utilizados (sistema operacional, tamanho de tela, versões etc). Não serão coletadas informações que permitam recuperar a identidade do participante, cujas respostas e dados serão salvos de maneira anônima; 
    \item Início do teste com áudio e tipografia. Em cada rodada, será apresentado um áudio contendo uma fala e sua transcrição. Em alguns casos, essa transcrição terá tido o desenho de suas letras modificado de acordo com nossa abordagem de extração e representação de prosódia. Em outros casos, essas modificações não corresponderão ao áudio. O participante deverá relacionar a tipografia que corresponde ao áudio, produzindo um índice de acerto que nos indicará a efetividade de nossa abordagem;
    \item Na segunda etapa e  e , haverá um conjunto de questões de escala de Likert, adaptadas de \cite{translatology}, visando mensurar medidas subjetivas como \textit{transporte} (questões como “enquanto fazia o teste da etapa anterior mantive plena consciência do ambiente em meu entorno”), \textit{identificação com as personagens} (questões como “as percepções e emoções das personagens estavam claras para mim”) e \textit{realismo percebido} (questões como “as personagens me pareceram realistas”). \todo{Paula, note que esta etapa não pensei para o primeiro teste e sim para o seguinte, com legendas. Mas eis o que me ocorreu: talvez possamos jogar umas questões de escala Likert mesmo no primeiro teste e correlacioná-las aos índices de performance dos participantes. Não responde ao nosso objetivo principal, mas não o influencia negativamente (seria ao final) e talvez traga algum elemento novo.}
\end{itemize}

Tanto no teste com quanto no sem supervisão, o participante pode abandonar o site a qualquer momento, sem prejuízo a si.

\todo{Paula: não estou falando sobre o conteúdo das frases para que possa usar um vídeo no segundo experimento (o das legendas) sem ter que escolher um filme agora. Você acha isso um problema?}

% As animações são frases sinalizadas em Libras traduzidas do livro de ciências do Ensino Fundamental I. As frases das duas primeiras etapas serão diferentes, além disso, para cada participante, o conjunto de frases será escolhido aleatoriamente.


% Likert
%As opções de respostas, concordância ou não para cada frase, serão:
%\begin{enumerate}
%\item Discordo totalmente
%\item Discordo parcialmente
%\item Indiferente
%\item Concordo parcialmente
%\item Concordo totalmente
%\end{enumerate}


\subsection{Plano de Recrutamento}

\noindent Recrutaremos participantes que cumpram os critérios de inclusão através de mensagem de e-mail e divulgação em redes sociais direcionadas a alunos de graduação, pós-graduação e funcionários da Unicamp.

Adicionalmente, serão afixados cartazes em quadros de aviso pela Unicamp.

%%%%%%%%%%
\section{Cronograma}
\noindent O projeto tem previsão de duração de 3 meses com início previsto após a aprovação do projeto no Comitê de Ética em Pesquisa, seguindo um plano de trabalho divido nas etapas apresentadas na Tabela~\ref{table1}, conforme cronograma apresentado.


\begin{table}[H]
\centering
\resizebox{\textwidth}{!}{%
\begin{tabular}{lll}
\textbf{Descrição Atividade} & \textbf{Período Planejado} & \textbf{Observações}  \\ \cline{1-3}
Recrutamento & 22/4/19 — 28/4/19 & \begin{tabular}[l]{@{}l@{}}Etapa de recrutamento dos \\ participantes da pesquisa.\end{tabular} \\ \hdashline[0.5pt/3pt]
Execução do Estudo & 29/4/19 — 26/5/19 & \begin{tabular}[l]{@{}l@{}}Avaliações supervisionadas e \\ período em que o site estará aberto.\end{tabular} \\ \hdashline[0.5pt/3pt]
Análise dos Resultados & 27/5/19 — 17/6/19 & \begin{tabular}[l]{@{}l@{}}Etapa de compilação \\ e análise dos resultados.\end{tabular} \\ \hdashline[0.5pt/3pt]
Publicação do Artigo & 18/6/19 — 15/7/19 & \begin{tabular}[l]{@{}l@{}}Etapa de registro dos resultados obtidos e\\ e escrita de artigo para divulgação dos \\ principais resultados.\end{tabular} \\ \hdashline[0.5pt/3pt]
\end{tabular}%
}
\caption{Cronograma do Projeto}
\label{table1}
\end{table}

\noindent O recrutamento para a participação na pesquisa só será iniciada efetivamente após aprovação do comitê de ética.

%%%%%%%%%%
\section{Orçamento}
\noindent A hospedagem do site será feita em servidores da empresa de infraestrutura web \textit{Surge}, que apresenta um plano gratuito sob o qual nossa demanda prevista se enquadra. O banco de dados com as respostas dos participantes ficará no serviço \textit{Firebase}, da empresa \textit{Google}, que também possui planos gratuitos para sites de menor complexidade, como o nosso.

Os recursos necessários para a realização da parte presencial e supervisionada da pesquisa, tais como mesas e computador, serão emprestados da Faculdade de Engenharia Elétrica e de Computação. Porém, teremos as despesas suplementares referente a pesquisa que estão detalhadas abaixo:

\begin{table}[h]
\caption{Detalhamento do Orçamento}
\centering
\vspace{0.5cm}
\begin{tabular}{l|r}
Identificação do Orçamento & Valor em Reais (R\$) \\
\hline 
Água        & 10,00 \\
Impressão de Instruções e Termos  & 15,00 \\
Impressão de Cartazes  & 25,00 \\
\hline 
Total & 50,00
\end{tabular}
\end{table}

%%%%%%%%%%
\section{Critérios de inclusão e exclusão dos participantes da pesquisa}

\noindent Para ser considerado elígivel o participante deverá ser fluente em português, alfabetizado e maior de idade. Adicionalmente, os participantes voluntários da etapa online sem supervisão deverão ter acesso a um computador com sistema de som, internet e versões recentes dos navegadores mais comuns.

Serão excluídos deste estudo indivíduos:

\begin{itemize}
\item indivíduos em situação de vulnerabilidade como pacientes em tratamento de doenças graves ou com problemas mentais;
\item esteja com algum tipo de debilidade no seu senso de movimento e equilíbrio (dores de cabeça, gripe, enxaquecas, ansiedade ou estar sob a influência de álcool e drogas);
\item possua alguma deficiência visual que impeça o reconhecimento das modificações visuais tipográficas propostas pelo estudo.
\end{itemize}

Distúrbios ou doenças oculares como daltonismo, ametropias ou nistagmo não são critérios excludentes, mas podem dificultar a interpretação dos estímulos visuais durante a pesquisa. Serão aceitos indivíduos que utilizam ou não lentes corretivas. 

O local onde a parte supervisionada dos estudos será realizada possui acesso para cadeirantes.

%%%%%%%%%%
\section{Riscos e Benefícios envolvidos na execução da pesquisa}
\noindent Tanto os participantes que fizerem o experimento sem supervisão em ambiente online quanto aqueles do grupo de controle que o fizerem supervisionados no \textsc{lca} da \textsc{feec} serão expostos aos áudios e imagens e responderão aos questionários em uma única sessão.

Todos os equipamentos utilizados neste experimento são seguros e indolores. 

Não há riscos previsíveis para os participantes da pesquisa.

Para garantir conforto e evitar cansaço, os participantes do grupo de controle terão acesso a cadeiras confortáveis e o ambiente estará adequadamente iluminado e climatizado. Além disso, os participantes serão monitorados frequentemente e, a qualquer momento, o usuário poderá sinalizar qualquer desconforto e a avaliação será interrompida imediatamente. 

Caso possíveis sintomas apresentados persistam após a interrupção da avaliação, o participante será encaminhado ao atendimento de saúde disponível no Pronto Socorro do Hospital das Clínicas da Unicamp, a 5 minutos do \textsc{lca}.

Não existem benefícios previstos para os participantes da pesquisa de forma individual. Porém, o resultado desta pesquisa visa trazer benefícios a todos que consome mídias audiovisuais com \textit{closed captions}.

%%%%%%%%%%
\section{Critérios de Encerramento ou Suspensão da Pesquisa}
\noindent O encerramento ou suspensão de exibição dos áudios e textos se dará caso o participante sofra qualquer tipo de desconforto ou decida encerrar a avaliação, sem que seja necessário nenhum motivo específico. 

%%%%%%%%%%
\section{Resultados do Estudo}
\noindent O estudo tem como resultado final a apresentação dos resultados de compreensão da associação entre áudio e modificações visuais na tipografia correspondentes, bem como de o quanto essa abordagem modifica medidas subjetivas auto-reportadas de engajamento com conteúdos audiovisuais.

%%%%%%%%%%
\section{Divulgação dos Resultados}
\noindent A divulgação deste estudo se dará em forma de artigo científico. Além disso, os resultados deste estudo farão parte de dissertação de mestrado na Faculdade de Engenharia Elétrica e de Computação (\textsc{feec}).

%%%%%%%%%%
\section{Requisitos Específicos do Protocolo de Pesquisa}
\noindent Não foram identificados outros requisitos específicos do protocolo de pesquisa além dos já mencionados nas seções anteriores.

\addcontentsline{toc}{section}{Referências Bibliográficas}

% \bibliographystyle{plain}  %% nome-ano %abnt
\bibliographystyle{abntex2-alf}
\bibliography{ref}



\newpage
\addcontentsline{toc}{section}{Anexo}
%\section*{Anexo}

%\subsection{Questões}

\end{document}


