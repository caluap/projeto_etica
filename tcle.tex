\documentclass[a4paper,11pt,titlepage,singlespacing]{article}
\usepackage[T1]{fontenc} 
\usepackage[utf8]{inputenc}
\usepackage[brazilian]{babel}

\usepackage[bottom=2cm,top=2cm,left=2cm,right=2cm]{geometry}
\usepackage{indentfirst}
\usepackage{soulutf8}
\usepackage{color}

\usepackage{fancyhdr}
\usepackage{lastpage}
\pagestyle{fancy}
\fancyhf{}
\rhead{}
\renewcommand{\headrulewidth}{0pt}
\lfoot{Rubrica do pesquisador:} 
\cfoot{Rubrica do participante:}
\rfoot{Página \thepage\ de \pageref{LastPage}}


%%%%%%% v-v-v-v-v-v-v %%%%%%%

\usepackage[easyscsl]{kpfonts}
\usepackage[T1]{fontenc}

\linespread{1.25}




\begin{document}

%%%%%%%%%%%%%%%%%%%


	\begin{center}
		\textbf{\Large{\textsc{termo de consentimento livre e esclarecido}}}\\
        \vspace{5pt}
        \huge\textit{A voz na letra: \\ avaliação de representações visuais \\ de prosódia em tipografia}\\
        \vspace{20pt}
        \large{Responsável: Profª Drª Paula Dornhofer Paro Costa}\\
        \vspace{10pt}
        Número do \textsc{caae}: \\
        \vspace{15pt}
	\end{center}

\noindent Você está sendo convidado a participar como voluntário de um projeto de pesquisa conduzido pelo Departamento de Engenharia de Computação e Automação Industrial (\textsc{dca}), da Faculdade de Engenharia Elétrica e de Computação (\textsc{feec}) da Universidade Estadual de Campinas (Unicamp).

Este documento, chamado \textit{Termo de Consentimento Livre e Esclarecido}, visa assegurar seus direitos e deveres como participante e é elaborado em duas vias. Uma ficará com você e outra com o pesquisador.

Por favor, leia com atenção e calma, aproveitando para esclarecer suas dúvidas. Se houver perguntas antes ou mesmo depois de assiná-lo, sinta-se à vontade para esclarecê-las com o pesquisador. Se preferir, pode levar este Termo para casa e consultar seus familiares ou outras pessoas antes de decidir participar. Não haverá nenhum tipo de penalização ou prejuízo se você não aceitar participar ou retirar sua
autorização em qualquer momento.


\section*{Justificativa e objetivos}

\noindent Há fortes indícios de que, para ler bem, um leitor precisa não apenas saber reconhecer cada letra mas também saber como encadear seus sons em ritmos e entonações fluentes — não basta ver, é preciso também ouvir, mesmo que dentro da própria mente. 

Em algumas situações esse processão não é fácil: uma pessoa que está aprendendo uma língua estrangeira não sabe de antemão como costumam soar suas frases; uma criança que está aprendendo a ler pode ter dificuldades em relacionar a língua que fala com os novos símbolos que lhe são apresentados; alguém que tenha uma deficiência auditiva pode nunca ter conhecido os sons por trás do alfabeto.

Nossa pesquisa busca lidar com essa situação, criando maneiras de aproximar aquilo que se vê (as letras) daquilo que se ouve (a fala). Assim, estamos explorando formas de manipular o desenho tipográfico de um texto de acordo com uma enunciação sonora do mesmo – atores e atrizes lendo em voz alta e esse som sendo transformado em imagem que aplicamos na tipografia.

Este estudo para o qual lhe convidamos a participar é a avaliação da inteligibilidade desses sinais gráficos nas letras — é possível saber quando derivam do som em oposição a quando são aleatórios? Ademais, queremos também entender se a presença dessas modulações visuais mudam de alguma como você se relaciona com uma dada obra — as emoções das personagens ficam mais claras? Você mergulha mais profundamente na obra? A obra em si parece mais realista?

\section*{Procedimentos}

\noindent Participando do estudo você está sendo convidado a participar de uma única sessão de avaliação a ser realizada nas dependências do Laboratório do Departamento de Engenharia de Computação e Automação Industrial (\textsc{lca}), na Faculdade de Engenharia Elétrica e de Computação (\textsc{feec}).

Você será acompanhado(a) o tempo todo pelos pesquisadores para que possa esclarecer suas dúvidas ou comunicar seus desconfortos.

A avaliação é dividida em três etapas. Na primeira etapa você poderá nos informar alguns dados a seu respeito — idade e grau de formação e, opcionalmente, sexo e área de atuação. Esses dados não estarão associados a seu nome e, como um todo, suas respostas ao teste serão registradas de maneira anônima.

Na segunda etapa, executaremos um arquivo contendo uma fala. Junto a essa fala apresentaremos sua transcrição. Você poderá notar que as formas tipográficas desse texto foram modificadas. A ideia é que você consiga relacionar quais variações visuais corresponderam aos sons de fala que acabou de ouvir e quais não estão relacionadas. Esse processo será repetido diversas vezes.

Na última etapa, haverá algumas perguntas para buscar desvendar os efeitos dessa combinação de áudio e texto modificado pela fala.

Nossa previsão é que a avaliação dure entre 15 e 30 minutos, mas fique à vontade para fazê-la no tempo em que achar necessário.



\section*{Desconfortos e riscos}

\noindent Não há riscos previsíveis para os participantes da pesquisa. Todos os equipamentos utilizados neste experimento são seguros e, nas condições de uso previstas, não causam nenhum tipo de dano à saúde. Para garantir conforto e evitar cansaço, os participantes terão acesso a cadeiras confortáveis e o ambiente estará adequadamente iluminado e climatizado. Em qualquer momento o participante poderá sinalizar algum desconforto e o experimento será interrompido imediatamente.

O indivíduo não deve participar deste estudo caso:

\begin{itemize}
\item esteja em tratamento de alguma doença grave ou apresente alguma doença mental;
\item esteja com algum tipo de debilidade no seu senso de movimento e equilíbrio (dores de cabeça, gripe, enxaquecas, ansiedade ou estar sob a influência de álcool e drogas);
\item possua alguma deficiência visual que impeça o reconhecimento de imagens com detalhes finos.
\end{itemize}

\section*{Benefícios}


Não existem benefícios previstos para os participantes da pesquisa de forma individual. Porém o resultado desta pesquisa visa auxiliar na alfabetização e aprendizagem de pessoas surdas.

\section*{Acompanhamento e assistência}

Após o término da pesquisa não está previsto nenhum acompanhamento de longo prazo do participante. Caso ocorra algum tipo de desconforto persistente devido à utilização do dispositivo de realidade virtual, a equipe responsável pela condução da pesquisa providenciará seu encaminhamento imediato ao atendimento emergencial de saúde mais próximo, sem ônus de qualquer espécie ao participante da pesquisa.

Nos termos da resolução N\textsuperscript{\underline{o}} 466 do Conselho Nacional da Saúde (CNS), de 12 de dezembro de 2012, item V.6, o participante receberá a assistência integral, de forma gratuita, pelo tempo que for necessário em caso de complicações e/ou danos decorrentes da pesquisa.


\section*{Sigilo e privacidade}

As filmagens realizadas durante a avaliação serão armazenadas em computadores de acesso controlado e restrito aos pesquisadores responsáveis desta pesquisa. As filmagens serão analisadas para que possamos comparar os resultados entre vários participantes.

Você tem a garantia de que sua identidade será mantida em sigilo e, salvo expressa autorização, nenhuma informação será dada a outras pessoas que não façam parte da equipe de pesquisadores. Na divulgação dos resultados desse estudo, seu nome jamais será citado.


\section*{Ressarcimento e Indenização}

A participação é voluntária e não serão ressarcidas despesas de transporte, alimentação e de nenhum outro tipo, pois a coleta de dados da pesquisa se dará no CEPRE com participantes voluntários que não se deslocaram para o local especificamente para a pesquisa mas que já se encontram em atendimento pelo centro.

As despesas de transporte serão ressarcidas ao participante somente em caso de coletas feitas em horários e dias fora da rotina do mesmo.

Os participantes serão assistidos imediatamente em caso de desconforto e, havendo necessidade, serão acionadas as unidades de emergência públicas disponíveis. 

Nos termos da resolução N\textsuperscript{\underline{o}} 466 do Conselho Nacional da Saúde (CNS), de 12 de dezembro de 2012, item V.7, os participantes da pesquisa que vierem a sofrer qualquer tipo de dano resultante de sua participação na pesquisa, previsto ou não no Termo de Consentimento Livre e Esclarecido, têm direito à indenização, por parte do pesquisador, do patrocinador e das instituições envolvidas nas diferentes fases da pesquisa.

\section*{Contato}

Em caso de dúvidas sobre a pesquisa, você poderá entrar em contato com o pesquisador Rodolfo Luis Tonoli ou Paula Dornhofer Paro Costa:

\textbf{Endereço}: Av. Albert Einstein, 400, Cidade Universitária “Zeferino Vaz”, Campinas (SP), CEP
13083865 – Faculdade de Engenharia Elétrica e de Computação da Unicamp.

\textbf{Contato telefônico}: (19) 98177-3786 / (19) 3521-0246

\textbf{E-mail}: rltonoli@gmail.com ou paula@fee.unicamp.br

Em caso de denúncias ou reclamações sobre sua participação e sobre questões éticas do estudo,
você poderá entrar em contato com a secretaria do Comitê de Ética em Pesquisa (CEP) da UNICAMP das 08:30hs às 11:30hs e das 13:00hs as 17:00hs na Rua: Tessália Vieira de Camargo, 126; CEP 13083-887 Campinas – SP; telefone (19) 3521-8936 ou (19) 3521-7187; e-mail: cep@fcm.unicamp.br.

\section*{O Comitê de Ética em Pesquisa \textsc{(cep)}}

O papel do CEP é avaliar e acompanhar os aspectos éticos de todas as pesquisas envolvendo seres humanos. A Comissão Nacional de Ética em Pesquisa (CONEP), tem por objetivo desenvolver a regulamentação sobre proteção dos seres humanos
envolvidos nas pesquisas. Desempenha um papel coordenador da rede de Comitês de
Ética em Pesquisa (CEPs) das instituições, além de assumir a função de órgão
consultor na área de ética em pesquisas.
\vspace{10pt}

%\newpage
\section*{Consentimento livre e esclarecido}

\hl{Após ter recebido esclarecimentos sobre a natureza da pesquisa, seus objetivos, métodos, benefícios
previstos e seus potenciais riscos, aceito participar e declaro estar
recebendo uma via original deste documento assinada pelo pesquisador e por mim, tendo todas as folhas
por nós rubricadas.}

\hl{Declaro também que estou ciente que serão realizadas filmagens minhas ou de meu dependente legal durante a participação na avaliação e assim, declaro que:}

\vspace{5pt}
\noindent Nome do(a) participante:\hrulefill\\
\noindent \hl{RG ou CPF do participante:}\hrulefill\\
\noindent Contato Telefônico (opcional):\hrulefill\\
E-mail (opcional):\hrulefill\\
\vspace{5pt}

\noindent\begin{tabular}{ll}
\makebox[5in]{\hrulefill} & \makebox[1.5in]{Data:\hrulefill}\\
\end{tabular}\\
(Assinatura do participante)\\


\vspace{10pt}

\section*{Responsabilidade do Pesquisador}

Asseguro ter cumprido as exigências da resolução 466/2012 CNS/MS e
complementares na elaboração do protocolo e na obtenção deste Termo de
Consentimento Livre e Esclarecido. Asseguro, também, ter explicado e fornecido uma
via deste documento ao participante. Informo que o estudo foi aprovado pelo CEP
perante o qual o projeto foi apresentado. Comprometo-me a utilizar o material e os
dados obtidos nesta pesquisa exclusivamente para as finalidades previstas neste
documento ou conforme o consentimento dado pelo participante.\\

\vspace{5pt}

\noindent\begin{tabular}{ll}
\makebox[5in]{\hrulefill} & \makebox[1.5in]{Data:\hrulefill}\\
\end{tabular}\\
(Assinatura do pesquisador)\\

%%%%%%%%%%%%%%%%%%%



\end{document}
