\documentclass[a4paper,11pt,titlepage,singlespacing]{article}
\usepackage[T1]{fontenc} 
\usepackage[utf8]{inputenc}
\usepackage[brazilian]{babel}

\usepackage[bottom=2cm,top=2cm,left=2cm,right=2cm]{geometry}
\usepackage{indentfirst}
\usepackage{soulutf8}
\usepackage{color}

\usepackage{fancyhdr}
\usepackage{lastpage}
\pagestyle{fancy}
\fancyhf{}
\rhead{}
\renewcommand{\headrulewidth}{0pt}
\lfoot{Rubrica do pesquisador:} 
\cfoot{Rubrica do participante:}
\rfoot{Página \thepage\ de \pageref{LastPage}}


%%%%%%% v-v-v-v-v-v-v %%%%%%%

\usepackage[easyscsl]{kpfonts}
\usepackage[T1]{fontenc}

\linespread{1.25}




\begin{document}

%%%%%%%%%%%%%%%%%%%


	\begin{center}
		\textbf{\Large{\textsc{termo de consentimento livre e esclarecido}}}\\
        \vspace{5pt}
        \huge\textit{A voz na letra: \\ avaliação de representações visuais \\ de prosódia em tipografia}\\
        \vspace{20pt}
        \large{Responsável: Profª Drª Paula Dornhofer Paro Costa}\\
        \vspace{10pt}
        Número do \textsc{caae}: \\
        \vspace{15pt}
	\end{center}

\noindent Você está sendo convidado a participar como voluntário de um projeto de pesquisa conduzido pelo Departamento de Engenharia de Computação e Automação Industrial (\textsc{dca}), da Faculdade de Engenharia Elétrica e de Computação (\textsc{feec}) da Universidade Estadual de Campinas (Unicamp).

Este documento, chamado Termo de Consentimento Livre e Esclarecido, visa assegurar seus direitos e deveres como participante e é elaborado em duas vias. Uma ficará com você e outra com o pesquisador.

Por favor, leia com atenção e calma, aproveitando para esclarecer suas dúvidas. Se houver perguntas antes ou mesmo depois de assiná-lo, sinta-se à vontade para esclarecê-las com o pesquisador. Se preferir, pode levar este Termo para casa e consultar seus familiares ou outras pessoas antes de decidir participar. Não haverá nenhum tipo de penalização ou prejuízo se você não aceitar participar ou retirar sua
autorização em qualquer momento.

\vspace{10pt}
\textbf{Justificativa e objetivos}:

Indivíduos surdos ou com perda auditiva severa, enfrentam diversos obstáculos durante a alfabetização. Materiais escolares geralmente são desenvolvidos para estudantes ouvintes, cuja primeira língua é o Português, e faltam professores bilíngues ou intépretes nas salas de aula. A alfabetização digital bilíngue, que considera e respeita a Libras como a primeira língua de estudantes surdos, pode reduzir tais obstáculos.

O presente estudo refere-se a uma das etapas de desenvolvimento do projeto TALES (Tecnologia Assistiva de Leitura para Surdos). Quando concluído, o TALES será uma ferramenta de tradução Português-Libras onde alunos surdos poderão visualizar o texto escrito em Português sinalizado em Libras \hl{por uma agente virtual no computador que fará a comunicação por meio de linguagem de sinais, denominada aqui como avatar}.

%por uma agente sinalizadora virtual, referenciada como avatar.

O principal objetivo deste estudo que você está sendo convidado(a) a participar é a avaliação da inteligibilidade da sinalização em Libras pelo avatar. \hl{Em outras palavras, o objetivo do estudo é verificar se o avatar está representando corretamente os sinais realizados, perguntando aos participantes se eles compreenderam, ou não, o conteúdo sinalizado pelo avatar.}
As animações serão apresentadas em um monitor convencional e também no Oculus Rift, que é um equipamento de realidade virtual, para fins de comparação da inteligibilidade dessas tecnologias.

\vspace{10pt}
\textbf{Procedimentos}:

Participando do estudo você está sendo convidado a participar de uma única sessão de avaliação do referido avatar sinalizador que será realizada nas dependências do Centro de Estudos e Pesquisas em Reabilitação Prof. Dr. Gabriel Porto (CEPRE).

Você será acompanhado(a) o tempo todo por um intérprete de Libras para que possa esclarecer suas dúvidas ou comunicar seus desconfortos.

A avaliação é dividida em três etapas. Na primeira etapa você assistirá vídeos curtos do avatar sinalizando frases em Libras em um monitor de computador convencional. Após a apresentação de cada frase, o intérprete perguntará a você se você conseguiu entender a frase e quais foram os sinais sinalizados.

Na segunda etapa, pediremos que você utilize um óculos de Realidade Virtual. Caso você nunca tenha usado um óculos desse tipo, fique à vontade para realizar perguntas sobre o mesmo. O óculos de realidade virtual é um dispositivo que vem sendo utilizado para entretenimento, para jogos e filmes e, após alguns ajustes iniciais, seu uso é totalmente seguro e não deve causar nenhum desconforto. Nesta etapa, você verá as frases sinalizadas em Libras no ambiente de Realidade de Virtual e o intérprete perguntará a você se você conseguiu entender as frases e quais foram os sinais sinalizados.

Na última etapa, faremos algumas perguntas a você, como numa entrevista, para saber sua opinião sobre o avatar e sobre o uso da tecnologia de realidade virtual.

Todas as etapas serão filmadas para que suas respostas possam ser comparadas com a de outros participantes posteriormente.

A avaliação deve durar aproximadamente 90 minutos.

\vspace{10pt}
\textbf{Desconfortos e riscos}:

\hl{Não há riscos previsíveis para os participantes da pesquisa. Todos os equipamentos utilizados neste experimento são seguros e, nas condições de uso previstas, não causam nenhum tipo de dano à saúde. Contudo, em alguns casos, o usuário imerso no ambiente de realidade virtual pode desenvolver tontura, náusea e dor de cabeça. Os participantes estarão sentados durante o experimento para redução desses efeitos e evitar quedas em caso de tontura. Em qualquer momento o usuário poderá sinalizar algum desconforto e o experimento será interrompido imediatamente.}

A avaliação foi projetada de modo que todos os possíveis desconfortos previsíveis advindos da realização desta avaliação fossem evitados ou minimizados.

Os participantes avaliarão o avatar sinalizador e responderão os questionários em uma única sessão a ser realizada no CEPRE.

O indivíduo não deve participar deste estudo caso:

\begin{itemize}
\item esteja em tratamento de alguma doença grave ou apresente alguma doença mental;
\item esteja com algum tipo de debilidade no seu senso de movimento e equilíbrio (dores de cabeça, gripe, enxaquecas, ansiedade ou estar sob a influência de álcool e drogas);
\item possua histórico de de convulsões ou epilepsia desencadeados por luzes ou padrões piscantes;
\item possua alguma deficiência visual que impeça o reconhecimento de animações digitais.
\end{itemize}

\vspace{10pt}
\textbf{Benefícios}:

Não existem benefícios previstos para os participantes da pesquisa de forma individual. Porém o resultado desta pesquisa visa auxiliar na alfabetização e aprendizagem de pessoas surdas.

\vspace{10pt}
\textbf{Acompanhamento e assistência}:

Após o término da pesquisa não está previsto nenhum acompanhamento de longo prazo do participante. Caso ocorra algum tipo de desconforto persistente devido à utilização do dispositivo de realidade virtual, a equipe responsável pela condução da pesquisa providenciará seu encaminhamento imediato ao atendimento emergencial de saúde mais próximo, sem ônus de qualquer espécie ao participante da pesquisa.

Nos termos da resolução N\textsuperscript{\underline{o}} 466 do Conselho Nacional da Saúde (CNS), de 12 de dezembro de 2012, item V.6, o participante receberá a assistência integral, de forma gratuita, pelo tempo que for necessário em caso de complicações e/ou danos decorrentes da pesquisa.

\vspace{10pt}
\textbf{Sigilo e privacidade}:

As filmagens realizadas durante a avaliação serão armazenadas em computadores de acesso controlado e restrito aos pesquisadores responsáveis desta pesquisa. As filmagens serão analisadas para que possamos comparar os resultados entre vários participantes.

Você tem a garantia de que sua identidade será mantida em sigilo e, salvo expressa autorização, nenhuma informação será dada a outras pessoas que não façam parte da equipe de pesquisadores. Na divulgação dos resultados desse estudo, seu nome jamais será citado.


\vspace{10pt}
\textbf{Ressarcimento e Indenização}:
	
A participação é voluntária e não serão ressarcidas despesas de transporte, alimentação e de nenhum outro tipo, pois a coleta de dados da pesquisa se dará no CEPRE com participantes voluntários que não se deslocaram para o local especificamente para a pesquisa mas que já se encontram em atendimento pelo centro.

As despesas de transporte serão ressarcidas ao participante somente em caso de coletas feitas em horários e dias fora da rotina do mesmo.

Os participantes serão assistidos imediatamente em caso de desconforto e, havendo necessidade, serão acionadas as unidades de emergência públicas disponíveis. 

Nos termos da resolução N\textsuperscript{\underline{o}} 466 do Conselho Nacional da Saúde (CNS), de 12 de dezembro de 2012, item V.7, os participantes da pesquisa que vierem a sofrer qualquer tipo de dano resultante de sua participação na pesquisa, previsto ou não no Termo de Consentimento Livre e Esclarecido, têm direito à indenização, por parte do pesquisador, do patrocinador e das instituições envolvidas nas diferentes fases da pesquisa.

\vspace{10pt}
\textbf{Contato}:

Em caso de dúvidas sobre a pesquisa, você poderá entrar em contato com o pesquisador Rodolfo Luis Tonoli ou Paula Dornhofer Paro Costa:

\textbf{Endereço}: Av. Albert Einstein, 400, Cidade Universitária “Zeferino Vaz”, Campinas (SP), CEP
13083865 – Faculdade de Engenharia Elétrica e de Computação da Unicamp.

\textbf{Contato telefônico}: (19) 98177-3786 / (19) 3521-0246

\textbf{E-mail}: rltonoli@gmail.com ou paula@fee.unicamp.br

Em caso de denúncias ou reclamações sobre sua participação e sobre questões éticas do estudo,
você poderá entrar em contato com a secretaria do Comitê de Ética em Pesquisa (CEP) da UNICAMP das 08:30hs às 11:30hs e das 13:00hs as 17:00hs na Rua: Tessália Vieira de Camargo, 126; CEP 13083-887 Campinas – SP; telefone (19) 3521-8936 ou (19) 3521-7187; e-mail: cep@fcm.unicamp.br.

\vspace{10pt}
\textbf{O Comitê de Ética em Pesquisa (CEP)}.

O papel do CEP é avaliar e acompanhar os aspectos éticos de todas as pesquisas envolvendo seres humanos. A Comissão Nacional de Ética em Pesquisa (CONEP), tem por objetivo desenvolver a regulamentação sobre proteção dos seres humanos
envolvidos nas pesquisas. Desempenha um papel coordenador da rede de Comitês de
Ética em Pesquisa (CEPs) das instituições, além de assumir a função de órgão
consultor na área de ética em pesquisas.
\vspace{10pt}

%\newpage
\textbf{Consentimento livre e esclarecido}:

\hl{Após ter recebido esclarecimentos sobre a natureza da pesquisa, seus objetivos, métodos, benefícios
previstos e seus potenciais riscos, aceito participar e declaro estar
recebendo uma via original deste documento assinada pelo pesquisador e por mim, tendo todas as folhas
por nós rubricadas.}

\hl{Declaro também que estou ciente que serão realizadas filmagens minhas ou de meu dependente legal durante a participação na avaliação e assim, declaro que:}

\vspace{5pt}
\noindent Nome do(a) participante:\hrulefill\\
\noindent \hl{RG ou CPF do participante:}\hrulefill\\
\noindent Contato Telefônico (opcional):\hrulefill\\
E-mail (opcional):\hrulefill\\
\vspace{5pt}

\noindent\begin{tabular}{ll}
\makebox[5in]{\hrulefill} & \makebox[1.5in]{Data:\hrulefill}\\
\end{tabular}\\
(Assinatura do participante)\\


\vspace{10pt}

\textbf{Responsabilidade do Pesquisador}:

Asseguro ter cumprido as exigências da resolução 466/2012 CNS/MS e
complementares na elaboração do protocolo e na obtenção deste Termo de
Consentimento Livre e Esclarecido. Asseguro, também, ter explicado e fornecido uma
via deste documento ao participante. Informo que o estudo foi aprovado pelo CEP
perante o qual o projeto foi apresentado. Comprometo-me a utilizar o material e os
dados obtidos nesta pesquisa exclusivamente para as finalidades previstas neste
documento ou conforme o consentimento dado pelo participante.\\

\vspace{5pt}

\noindent\begin{tabular}{ll}
\makebox[5in]{\hrulefill} & \makebox[1.5in]{Data:\hrulefill}\\
\end{tabular}\\
(Assinatura do pesquisador)\\

%%%%%%%%%%%%%%%%%%%



\end{document}
